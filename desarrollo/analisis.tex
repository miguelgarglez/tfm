\subsection{Análisis de requisitos\label{SEC:REQUISITOS}}

En esta sección de análisis, se detallarán los requisitos funcionales y no funcionales de la aplicación.

\paragraph{\textbf{Requisitos Funcionales}}

Expondremos los requisitos funcionales de la aplicación, definiendo lo que esperamos que nuestra aplicación desarrollada 
cumpla. Estos requisitos se han obtenido a partir de la motivación del proyecto y de las funcionalidades que ofrece la API
de Spotify. A continuación, se detallan los requisitos funcionales de la aplicación:

\begin{functional}
    \item El sistema debe permitir que los usuarios hagan \textit{login} a través de Spotify.
    \begin{functional}
        \item El sistema hará uso del servicio de autenticación de Spotify.
    \end{functional}
    \item El sistema debe permitir que los usuarios obtengan información de sus cuentas de Spotify.
    \item El sistema debe permitir que múltiples usuarios estén dados de alta (hayan hecho \textit{login} con su cuenta de Spotify) al 
    mismo tiempo en la aplicación.
    \item El sistema debe ser capaz de comunicarse con la API de Spotify, y recibir y enviar información con éxito al servicio.
    \item El sistema debe ser capaz de obtener los artistas o canciones más escuchados de los usuarios en el corto, medio o largo plazo.
    \begin{functional}
        \item El sistema obtendrá hasta 50 artistas o canciones más escuchados de los usuarios.
    \end{functional} 
    \item El sistema debe ser capaz de ofrecer recomendaciones de canciones a los usuarios basándose en canciones, géneros o artistas que seleccionen los usuarios.
    \begin{functional}
        \item El sistema debe permitir a los usuarios seleccionar canciones, géneros o artistas para obtener recomendaciones.
        \item El sistema recomendará hasta 50 canciones a los usuarios.
    \end{functional} 
    
    \item El sistema debe permitir que múltiples usuarios dados de alta en la aplicación creen una playlist colaborativa.
    \begin{functional}
        \item El sistema debe permitir a los usuarios elegir la duración de la playlist colaborativa.
        \item El sistema debe ofrecer distintas estrategias de agregación para la selección de canciones en la playlist colaborativa.
        \item El sistema debe mostrar la playlist generada en pantalla.
        \item El sistema debe permitir a los usuarios guardar la playlist colaborativa en sus cuentas de Spotify.
    \end{functional} 
    
\end{functional}

\paragraph{\textbf{Requisitos No Funcionales}}

A continuación, se detallan los requisitos no funcionales de la aplicación:

\begin{nonfunctional}
    \item \textbf{Rendimiento}: La aplicación debe responder a las solicitudes del usuario en menos de 3 segundos bajo condiciones normales de carga.
    \item \textbf{Disponibilidad}: La aplicación debe estar disponible el 99\% del tiempo, asegurando el acceso de los usuarios a sus funcionalidades.
    \item \textbf{Compatibilidad}: La aplicación debe ser compatible con las versiones más recientes de navegadores web populares como Chrome, Firefox, Safari y Edge.
    \item \textbf{Mantenibilidad}: El código de la aplicación debe seguir las buenas prácticas de programación, siendo legible, bien documentado y fácil de mantener.
    \item \textbf{Usabilidad}: La interfaz de usuario debe ser intuitiva y fácil de usar, asegurando que los usuarios puedan realizar todas las operaciones sin necesidad de instrucciones adicionales.
\end{nonfunctional}

\subsection{Diagramas de casos de uso principales\label{SEC:CASOS_USO}}

A continuación, se presentan los diagramas de casos de uso principales de la aplicación. Estos diagramas representan las interacciones en las cuáles
se sustenta la funcionalidad de la aplicación.

\begin{figure}[Diagramas de casos de uso principales de la aplicación]{FIG:USE_CASE_DIAGRAM}
    {Diagramas de casos de uso principales de la aplicación}
    \subfigure[]{\textit{Login}}{\image{5.7cm}{}{propias/login_usecase.png}} \quad
    \subfigure[]{\textit{Crear Playlist}}{\image{8cm}{}{propias/create_playlist_usecase.png}}
\end{figure}

\newpage
