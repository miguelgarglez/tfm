En esta sección se expondrá el diseño de la interfaz de usuario, así como el de la arquitectura.

\subsection{Diseño de la interfaz de usuario\label{SEC:DISENO_INTERFAZ}}

Véanse en la Figura \ref{FIG:MAQUETAS} las maquetas de la interfaz de usuario de la aplicación que hemos tenido en mente a la hora de llevar a cabo
la implementación de nuestra aplicación.

\begin{figure}[Maquetas de la aplicación a desarrollar]{FIG:MAQUETAS}
    {Maquetas de la aplicación a desarrollar}
          \image{14cm}{}{propias/maquetas.png}
\end{figure}


\subsection{Arquitectura del sistema\label{SEC:DISENO_ARQUITECTURA}}

La arquitectura de la aplicación se basa en una arquitectura cliente-servidor, donde el cliente es el \textit{frontend} que hemos desarrollado, que corre en el navegador web, y 
el servidor es la API de \textit{Spotify}.La aplicación se ejecuta en el lado del cliente (en el navegador) y utiliza opciones de almacenamiento locales para mantener la seguridad y 
la integridad de los tokens de acceso a la API de \textit{Spotify}. 

La arquitectura del sistema se compone de los siguientes elementos clave:

\begin{itemize}
    \item \textbf{\textit{Frontend}}: Interfaz de usuario desarrollada con \textit{Flutter} que ofrece una experiencia de usuario interactiva 
    y dinámica. Se encarga de presentar la información, recoger las interacciones del usuario y comunicarse con la API de Spotify.
    \item \textbf{Almacenamiento Local}: Utilizado para guardar de forma segura los tokens de acceso necesarios para la interacción con la API de \textit{Spotify}
    y otros datos necesarios para el funcionamiento de la aplicación, asegurando que la información sensible se maneje correctamente. Utilizaremos
    la API de \textit{Hive} \cite{hive} para su implementación.
    \item \textbf{API de \textit{Spotify}}: Servicio externo que proporciona acceso a los datos de música, artistas, y \textit{playlists}, así como las funcionalidades 
    para crear y modificar playlists en las cuentas de los usuarios.
\end{itemize}

La interacción entre los componentes se puede observar en la Figura \ref{FIG:SEQUENCE_DIAGRAM} en un diagrama de secuencia
y se puede resumir en los siguientes pasos:

\begin{enumerate}
    \item El \textbf{\textit{Frontend}} envía solicitudes a la API de Spotify para obtener datos de música o realizar operaciones relacionadas con playlists.
    \item La \textbf{API de \textit{Spotify}} responde a estas solicitudes, proporcionando los datos o confirmando la realización de las operaciones solicitadas.
    \item Durante este proceso, el \textbf{Almacenamiento Local} se utiliza para guardar y recuperar los tokens de acceso y otros datos relevantes, asegurando 
    que las solicitudes a la API de Spotify estén autenticadas y sean seguras.
\end{enumerate}


\begin{figure}[Diagrama de secuencia de interacción con la API de \textit{Spotify}]{FIG:SEQUENCE_DIAGRAM}
    {Diagrama de secuencia de interacción con la API de \textit{Spotify}}
          \image{14cm}{}{propias/interaccion_api.png}
\end{figure}



\subsubsection{Frontend\label{subsec:frontend}}

El \textit{framework} de desarrollo de aplicaciones \textit{Flutter} \cite{flutter} ha sido el elegido para el desarrollo del \textit{frontend} de la aplicación. Ya hemos
comentado en la Sección \ref{SEC:FLUTTER} que \textit{Flutter} es un \textit{framework} basado en el lenguaje de programación \textit{Dart}, cuya base se sienta
en los \textit{widgets}. Vamos a explicar cada uno de los actores principales en el desarrollo del \textit{frontend} de la aplicación a continuación.

\paragraph{Pantallas}

Tal y como se ha diseñado el \textit{frontend} de la aplicación, cada pantalla de la aplicación se ha desarrollado como un \textit{widget} independiente, 
el cual, a su vez está formado por widgets más pequeños que se combinan para formar la interfaz de usuario. Los módulos que definen las pantallas se agrupan en \textit{lib/screens}:

\begin{itemize}
  \item \texttt{generate\_playlist.dart}
  \item \texttt{get\_recommendations.dart}
  \item \texttt{get\_top\_items.dart}
  \item \texttt{item\_display.dart}
  \item \texttt{landing.dart}
  \item \texttt{playlist\_detail.dart}
  \item \texttt{playlist\_display.dart}
  \item \texttt{save\_playlist.dart}
  \item \texttt{track\_detail.dart}
  \item \texttt{user\_detail.dart}
  \item \texttt{users\_display.dart}
\end{itemize}

 % QUEDA COMENTAR POR AQUI

\paragraph{Widgets}

Los \textit{widgets} personalizados (la mayoría de \textit{widgets} utilizados en el desarrollo han sido propios del \textit{framework} o paquetes de \textit{pub.dev} \cite{pub_dev}) que 
se utilizan para formar las pantallas se agrupan en \textit{lib/widgets}:

\begin{itemize}
  \item \texttt{artist\_tile.dart}
  \item \texttt{expandable\_fab.dart}
  \item \texttt{track\_tile.dart}
\end{itemize}

 % QUEDA COMENTAR POR AQUI

\paragraph{Servicios}

Por otro lado, agrupados en \textit{lib/services}, se encuentran los módulos que se encargan de realizar las llamadas a la API de \textit{Spotify} y procesar las respuestas para uniformizarlas
y hacerlas más fáciles de manejar en el \textit{frontend}. En la Sección \ref{SEC:IMPLEMENTACION} se ahondará más en el uso de estos módulos. También en este mismo grupo se encuentra el módulo
de recomendación de canciones, que se ha desarrollado para llevar a cabo la agregación de las canciones recomendadas por \textit{Spotify} para los usuarios que forman un grupo. 
En este grupo tenemos:

\begin{itemize}
  \item \texttt{requests.dart}
  \item \texttt{recommendator.dart}
  \item \texttt{statistics.dart}
\end{itemize}

 % QUEDA COMENTAR POR AQUI

\paragraph{Modelos}

Cuando hablamos de uniformizar las respuestas de la API de \textit{Spotify}, nos referimos a que todas ellas se reducirán a un modelo de objeto determinado, para así uniformizar el manejo de
errores en las pantallas y las peticiones exitosas. Se han agrupado en \textit{lib/models} los modelos de objetos que se utilizan en la aplicación:

\begin{itemize}
  \item \texttt{artist.dart}
  \item \texttt{my\_response.dart}
  \item \texttt{track.dart}
  \item \texttt{user.dart}
\end{itemize}

 % QUEDA COMENTAR POR AQUI


\subsubsection{Almacenamiento Local\label{subsec:almacenamiento_local}}

El almacenamiento local se ha llevado a cabo con \textit{Hive} \cite{hive} para \textit{Flutter}. Ha sido utilizada
para guardar de forma segura a los usuarios que hagan login en la aplicación, junto con sus respectivos tokens de
 acceso a la API de \textit{Spotify}. Este ha sido el principal uso que se le ha dado, pero también se ha utilizado 
 para guardar pequeñas variables de forma temporal.

 % QUEDA COMENTAR POR AQUI



\subsubsection{API de Spotify\label{subsec:api_spotify_diseno}}

En esta sección realizaremos una visión general de la API de \textit{Spotify} \cite{spotify_api}, e iremos pasando por
los \textit{endpoints} más destacados, que serán útiles para el desarrollo de la aplicación. No obstante, 
en la Sección \ref{SEC:IMPLEMENTACION} se ahondará más en el uso de cada uno de ellos.

\paragraph{Visión general de la API\label{subsec:vision_general_api}}

La API de \textit{Spotify} es una API \textit{REST}, que utiliza el formato \textit{JSON} para el intercambio 
de datos. Es importante destacar que para hacer uso de esta API, es necesario darse de alta como desarrollador, asumiendo que se 
tiene una cuenta de \textit{Spotify}. Una vez hecho esto, se podrán dar de alta aplicaciones en la cuenta de desarrollador desde el
\textit{Dashboard}, y tener así los datos y credenciales necesarios para hacer uso de la API. 
En el \textit{Dashboard} se pueden ver las aplicaciones dadas de alta, así como crear nuevas. Además, se puede entrar en cada
aplicación para ver datos de utilización de \textit{endpoints} y editar ajustes sobre ella, como la dirección del sitio web,
las posibles \textit{redirect URIs} para el flujo de autenticación, etc.
Obsérvese la Figura \ref{FIG:SPOTIFY_DEV} a modo de orientación.

\begin{figure}[Visión general \textit{Spotify for Developers}]{FIG:SPOTIFY_DEV}
    {Visión general \textit{Spotify for Developers} \\
    {\scriptsize Imágenes extraídas de \href{https://developer.spotify.com/dashboard/}{developer.spotify.com}}}
    \subfigure[]{\textit{Dashboard}}{\image{7cm}{}{propias/dashboard.png}} \quad
    \subfigure[]{Información aplicación}{\image{7cm}{}{propias/app_basic_info.png}} \quad
    \subfigure[]{Editar ajustes aplicación}{\image{7cm}{}{propias/app_settings.png}}
\end{figure}

Para hacerse una idea inicial de qué tipos de \textit{endpoints} nos harán falta, se puede hacer una primera aproximación
a la aplicación que se quiere desarrollar. Vamos a suponer que un grupo de usuarios de \textit{Spotify} está 
reunido e interesado en crear una playlist común que combine sus gustos musicales para escuchar en ese momento
(o en un futuro). Teniendo en cuenta esta situación, y el resultado que se quiere obtener, podemos
listar qué tipos de \textit{endpoints} o interacciones con la API nos harán falta:

\begin{itemize}
  \item \textbf{Datos de cuenta de los usuarios}, es decir, necesitaremos que los usuarios hagan 
  \textit{login} con sus cuentas de \textit{Spotify}, para mostrar su información básica, como su foto de
   perfil y nombre de usuario, y obtener un token de acceso para poder hacer posteriores llamadas a la API.
  \item \textbf{Canciones más escuchadas} porque es necesario saber sobre los gustos de los usuarios y,
  obtener las canciones más escuchadas de cada uno de los usuarios que forman el grupo, parece una buena forma de hacerlo.
  \item \textbf{Recomendaciones}, porque así, podremos aprovecharnos de las estrategias de recomendación 
  individuales de \textit{Spotify}, y obtener recomendaciones de canciones que puedan gustar a los usuarios 
  que forman el grupo, para así generar la playlist común.
  \item \textbf{Crear playlist}, porque así los usuarios podrán guardar en sus cuentas de \textit{Spotify} la
  playlist que se haya generado.
\end{itemize}

Todos estos \textit{endpoints} están disponibles en la API de \textit{Spotify}, cuya URL base es "textit{"https://api.spotify.com/v1}". Vamos a hacer un recorrido por cada uno de ellos.

\paragraph{Datos de cuenta de los usuarios\label{subsec:datos_cuenta_usuarios}}

Para que los usuarios puedan autenticarse en la aplicación con sus cuentas de \textit{Spotify}, y así poder crear playlists 
combinadas con su grupo, se llevará a cabo el flujo de autenticación PKCE 
(\textit{Proof Key for Code Exchange}) definido en \cite{spotify_pkce}, el cual es el recomendado por 
\textit{Spotify} para aplicaciones móviles u aplicaciones web de una sola página, donde la clave secreta
del cliente no puede ser protegida de manera segura. Se puede ver un gráfico del flujo en la Figura \ref{FIG:SPOTIFY_PKCE_FLOW}.

\begin{figure}[Flujo de autenticación PKCE]{FIG:SPOTIFY_PKCE_FLOW}
    {Flujo de autenticación PKCE}
          \image{14cm}{}{propias/flujo_pkce.png}
\end{figure}

Para llevar a cabo esta autenticación, la app tendrá que lanzar la URL de autenticación de \textit{Spotify}
("https://accounts.spotify.com/authorize"). Se explicará más en detalle la implementación de este flujo de autenticación
en la Sección \ref{SEC:FLUJO_AUTENTICACION}.

Una vez se lleva a cabo la autenticación del usuario con su cuenta de \textit{Spotify}, ya podemos hacer llamadas a la API
gracias al token de acceso que hemos obtenido. Comenzaremos por obtener los datos de la cuenta del usuario, como su foto de perfil, o su
nombre de usuario. Para ello, \textit{Spotify} nos ofrece un \textit{endpoint} llamado \textit{Get Current User's Profile} \cite{current_user_profile} cuya 
ruta es \texttt{/me}, \texttt{GET} es el método a utilizar, y no se necesitan parámetros adicionales.

\paragraph{Canciones más escuchadas\label{subsec:canciones_mas_escuchadas}}

Obtener las canciones más escuchadas por los usuarios será un factor fundamental para la generación de \textit{playlists}, descubrimos que hay un 
\textit{endpoint} que nos puede devolver los datos que necesitamos. En la documentación de la API \cite{spotify_api} está definido como 
\textit{Get User's Top Items} \cite{top_items}. Y es que, resulta que podremos obtener tanto las canciones más escuchadas, como los artistas más escuchados. 
Veremos cómo haremos uso de este \textit{endpoint} para la implementación de nuestra app en la Sección \ref{SEC:IMPLEMENTACION}. 

Como se puede observar en la Figura \ref{FIG:TOP_ITEMS}, \texttt{GET} es el método a utilizar y se especifican los siguientes parámetros:

\begin{figure}[Obtener canciones o artistas más escuchados]{FIG:TOP_ITEMS}
  {Obtener canciones o artistas más escuchados \\
  {\scriptsize Imagen extraída de \href{https://developer.spotify.com/documentation/web-api/reference/get-users-top-artists-and-tracks}{developer.spotify.com}}}
        \image{11cm}{}{propias/top_items.png}
\end{figure}

\begin{itemize}
  \item \texttt{type}: para indicar si queremos obtener artistas o canciones.
  \item \texttt{time\_range}: el período de tiempo sobre el que queremos obtener la lista \textit{items}.
  \item \texttt{limit}: el número de \textit{items} que querríamos obtener.
  \item \texttt{offset}: por si quisieramos obtener la lista a partir de una determinada posición.
\end{itemize}


\paragraph{Recomendaciones\label{subsec:recomendaciones}}

Para obtener recomendaciones de canciones, \textit{Spotify} nos ofrece un \textit{endpoint} llamado \textit{Get Recommendations} \cite{recommendations}.
En la Sección \ref{SEC:IMPLEMENTACION} ahondaremos más en detalle sobre cómo se ha hecho uso de este 
\textit{endpoint} para la implementación de la app. En la Figura \ref{FIG:RECOMMENDATIONS}, además de que
\texttt{GET} es el método a utilizar, se pueden observar cinco parámetros que se pueden especificar:

\begin{itemize}
  \item \texttt{limit}: el número de canciones que se quieren obtener.
  \item \texttt{market}: el mercado en el que deben estar presentes las canciones recomendadas,
  si se quisiera restringir.
  \item \texttt{seed\_artists}: una lista de identificadores de artistas, a partir de los cuales, 
  \textit{Spotify} obtendrá canciones para recomendar.
  \item \texttt{seed\_genres}: una lista de géneros, que funciona igual que la anterior.
  \item \texttt{seed\_tracks}: una lista de identificadores de canciones, que funciona igual que las anteriores.
\end{itemize}

Para este \textit{endpoint}, existen más parámetros opcionales que se pueden especificar, relacionados con 
la musicalidad, la energía, el tempo, la popularidad de las canciones, etc. Realmente tienen mucho potencial para 
la obtención de recomendaciones, pero no se han utilizado en la aplicación desarrollada. No obstante, pueden ser algo 
muy a tener en cuenta para trabajo futuro, y lo comentaremos en el Capítulo \ref{CAP:CONCLUSIONES}.

\begin{figure}[Obtener recomendaciones]{FIG:RECOMMENDATIONS}
    {Obtener recomendaciones \\
    {\scriptsize Imagen extraída de \href{https://developer.spotify.com/documentation/web-api/reference/get-recommendations}{developer.spotify.com}}}
          \image{11cm}{}{propias/recommendations.png}
\end{figure}

\paragraph{Crear playlist\label{subsec:crear_playlist}}

Para crear una playlist, \textit{Spotify} nos ofrece un \textit{endpoint} llamado \textit{Create Playlist} \cite{create_playlist}.
Como con los anteriores, ahondaremos más en cómo es utilizado en el capítulo \ref{CAP:DESARROLLO}. En la Figura \ref{FIG:CREATE_PLAYLIST}, además de
\texttt{POST} es el método a utilizar, se pueden observar los parámetros a especificar en el cuerpo de la petición:

\begin{itemize}
  \item \texttt{name}: el nombre de la playlist.
  \item \texttt{public}: si la playlist es pública o no. Es decir, si otros usuarios de \textit{Spotify} podrán verla o no.
  \item \texttt{collaborative}: si la playlist es colaborativa o no. Es decir, si se creará una \textit{playlist} con la capacidad
  de que se le añadan canciones por parte de otros usuarios, o no.
  \item \texttt{description}: la descripción de la playlist.
\end{itemize}

\begin{figure}[Crear playlist]{FIG:CREATE_PLAYLIST}
    {Crear playlist \\
    {\scriptsize Imagen extraída de \href{https://developer.spotify.com/documentation/web-api/reference/create-playlist}{developer.spotify.com}}}
          \image{11cm}{}{propias/create_playlist.png}
\end{figure}

Con este \textit{endpoint} se lleva a cabo la creación de la \textit{playlist}, que se guardará en la cuenta de \textit{Spotify} del usuario que
se elija. Pero aún faltaría añadir las canciones a la \textit{playlist} recién creada. Para ello, \textit{Spotify} nos ofrece un \textit{endpoint} llamado
\textit{Add Items to Playlist} \cite{add_to_playlist}. En la Figura \ref{FIG:ADD_TO_PLAYLIST}, además de observar que \texttt{POST} es el método a utilizar,
se pueden observar de nuevo los parámetros a especificar en el cuerpo de la petición:

\begin{itemize}
  \item \texttt{playlist\_id}: el identificador de la \textit{playlist} a la que se quieren añadir las canciones.
  \item \texttt{position}: la posición en la que se quieren añadir las canciones.
  \item \texttt{uris}: una lista de identificadores de canciones, que se añadirán a la \textit{playlist}.
\end{itemize}

Debe observarse además, que las \texttt{uris} se pueden pasar en la URL de la petición si no son demasiadas;
si no, se recomienda pasarlas en el cuerpo de la petición.

\begin{figure}[Añadir canciones a \textit{playlist}]{FIG:ADD_TO_PLAYLIST}
    {Añadir canciones a \textit{playlist} \\
    {\scriptsize Imagen extraída de \href{https://developer.spotify.com/documentation/web-api/reference/add-tracks-to-playlist}{developer.spotify.com}}}
          \image{11cm}{}{propias/add_to_playlist.png}
\end{figure}


\paragraph{Resumen de los \textit{endpoints} a utilizar\label{subsec:resumen_endpoints}}

En la tabla \ref{TB:ENDPOINTS} se resume la información de los \textit{endpoints} de la API de \textit{Spotify} que se han comentado en esta sección.


\begin{table}[Tabla resumen de los \textit{endpoints} a utilizar]{TB:ENDPOINTS}{Tabla resumen de los \textit{endpoints} de la API de \textit{Spotify} a utilizar.}
    \begin{tabular}{ c | c | c | c }
      \hline
      \multicolumn{4}{c}{\textbf{URL Base}} \\
      \hline
      \multicolumn{4}{c}{https://api.spotify.com/v1} \\
      \hline \hline
      \textbf{Nombre} & \textbf{Método} & \textbf{\textit{Auth Scopes}} & \textbf{Ruta} \\
      \hline \hline
      Get Current User's Profile & GET & \makecell{user-read-private \\ user-read-email} & /me \\
      \hline
      Get User's Top Items & GET & user-top-read & /me/top/\{type\} \\
      \hline
      Get Recommendations & GET & - & /recommendations \\
      \hline
      Get Available Genre Seeds & GET & - & /recommendations/available-genre-seeds \\
      \hline
      Create Playlist & POST & \makecell{playlist-modify-public \\ playlist-modify-private} & /users/\{user\_id\}/playlists \\
      \hline
      Add Items to Playlist & POST & \makecell{playlist-modify-public \\ playlist-modify-private} & /playlists/\{playlist\_id\}/tracks \\
      \hline
      Get Playlist & GET & - & /playlists/\{playlist\_id\} \\
      \hline
    \end{tabular}
  \end{table}
    