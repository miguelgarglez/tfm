En este capítulo se hablará sobre pruebas con los usuarios, y sus resultados. Puede que también
se hable de las pruebas que se hicieron para decidir qué estrategias de
agregación se iban a comparar.

\section{Estudios con usuarios\label{SEC:ESTUDIOS_USUARIOS}}

Será la forma en la que trataremos de evaluar nuestro sistema de recomendación, elaborando encuestas para recoger las sensaciones y satisfacción
que las \textit{playlists} generadas por nuestro sistema de recomendación han causado en los usuarios. Compararemos estrategias de agregación para tratar de sacar conclusiones
sobre que estrategias tienen mejor acogida.

\subsection{Gestión de usuarios\label{SEC:GESTION_USUARIOS}}

Como comentábamos en la Sección \ref{subsec:vision_general_api}, \textit{Spotify} nos permite dar de alta hasta 25 usuarios en el modo de desarrollo. Por ello, 

\section{Evaluaciones \textit{offline}\label{SEC:EVALUACIONES_OFFLINE}}

Aunque no con el mismo propósito, también hemos hecho uso de estas evaluaciones \textit{offline} para comparar distintas estrategias de agregación. Así seremos
capaces de ver cuáles se comportaban de manera más similar a otras estrategias, y cuáles se comportaban de manera más diferente. Esto nos dará la posibilidad de suprimir
estrategias que sean demasiado similares, para así centrar las comparaciones con usuarios en las estrategias que más diferencias presenten entre sí.
