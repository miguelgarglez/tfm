Elaboraremos encuestas para recoger las sensaciones y satisfacción
que las \textit{playlists} generadas por nuestro sistema de recomendación han causado en los usuarios. Compararemos estrategias de agregación para tratar de sacar conclusiones
sobre qué estrategias tienen mejor acogida, trataremos de sacar una puntuación de usabilidad para nuestra aplicación 
con una encuesta \textit{SUS} (System Usability Scale) y, por último, trataremos de sacar conclusiones sobre posibles nuevas funcionalidades que podrían ser interesantes 
para implementar en el futuro.

\subsection{Gestión de usuarios\label{SEC:GESTION_USUARIOS}}

Como comentábamos en la Sección \ref{subsec:vision_general_api}, \textit{Spotify} nos permite dar de alta hasta 25 usuarios en el modo de desarrollo. Esto sin duda 
ha supuesto una limitación para nuestras pruebas con usuarios, puesto que nosotros teníamos que saber qué personas
iban a probar nuestra aplicación para poder darlos de alta previamente. Por esto no hemos podido compartir nuestra aplicación y nuestras
encuestas con un número elevado de personas, lo que sin duda hubiera sido lo ideal, para tener el mayor número de opiniones posibles, junto con mayor 
variedad de grupos de personas. Nuestra encuesta ha tenido numerosas preguntas, las cuales tratarán de obtener:

\begin{itemize}
    \item Contexto sobre el grupo de personas que ha generado una \textit{playlist} con nuestra aplicación.
    \item Comparación de estrategias de agregación.
    \item Puntuación de usabilidad de la aplicación.
    \item Posibles nuevas funcionalidades que podrían ser interesantes para implementar en el futuro.
\end{itemize}

\subsection{Procedimiento de las pruebas con los usuarios\label{SEC:PROCEDIMIENTO}}

Las pruebas con usuarios se han llevado a cabo de manera presencial, reuniéndonos con los usuarios, dándolos de alta en nuestra aplicación para que pudieran utilizarla y 
explicándoles los pasos a seguir:

\begin{itemize}
    \item En primer lugar, que todos hicieran login en nuestra aplicación con su cuenta de \textit{Spotify}.
    \item Una vez todos aparecieran en la pantalla principal de la aplicación, les explicábamos que tenían que generar una \textit{playlist} con nuestra aplicación, eligiendo
    la opción: \textit{Comparar estrategias A y B}, e indicando una duración de 1 hora para la \textit{playlist}.
    \item Una vez generadas las \textit{playlist}, les pedíamos que revisaran las canciones que habían sido añadidas a las \textit{playlists}, que se quedaran con la 
    sensación que les transmitían, o si una le gustaba más que otra.
    \begin{itemize}
        \item Es importante destacar que, en este paso, si nuestro sistema de recomendación no era capaz de obtener una \textit{playlist}
        diferente para cada estrategia, solo se mostraría una en pantalla, y esta opción ha sido barajada en el formulario.
    \end{itemize}
    \item Compartíamos con ellos el formulario vía \textit{link}.
    \item Por último, les pedíamos que rellenaran el formulario de manera totalmente sincera.
\end{itemize}

Otros datos a tener en cuenta:

\begin{itemize}
    \item Hemos realizado las pruebas con 15 personas, de las cuales 8 han sido parejas, un grupo de 3 y otro de 4 personas.
    \item Hemos realizado las pruebas en un entorno controlado, en el que hemos podido explicar a los usuarios el procedimiento a seguir, y hemos podido resolver
    cualquier duda que les surgiera.
    \item La estrategia \textit{Multiplicative} ha sido expuesta como \textit{Estrategia A} y la estrategia \textit{Least Misery} como \textit{Estrategia B}, tanto en la 
    aplicación, como en el formulario.
\end{itemize}

\subsection{Contexto sobre el grupo de personas\label{SEC:CONTEXTOS}}

En nuestra encuesta, asignamos un nombre identificativo a cada grupo, preguntamos de cuántas personas se componía, 
y si consideraban si el grupo era homogéneo o heterogéneo. De estas preguntas, obtenemos lo siguiente:

\begin{itemize}
    \item \textbf{Número de personas en el grupo}: La mayoría de grupos han sido parejas, dejando otro grupo de 3 y de 4 personas.
    \item \textbf{Nombre para identificar el grupo}: Les pedíamos que pusieran los nombres de las personas que habían generado la \textit{playlist} en orden alfabético.
    \item \textbf{Homogeneidad del grupo}: Las parejas han coincidido en su percepción de la homogeneidad de sus gustos. Pero en los grupos
    más numerosos, ha habido opiniones dispares.
\end{itemize}

Sobre la valoración de la homogeneidad en las parejas, sí que creemos que nos pueden servir para sacar algunas
conclusiones sobre las teorías que empezamos a plantear en la Sección \ref{SEC:DECISIONES_TRAS_RESULTADOS}, como la de que 
un bajo número de personas en el grupo, (como las parejas), y la baja similitud en los gustos, podía provocar que 
las estrategias de agregación no aportaran diferencias en las recomendaciones.

\subsection{Comparación de estrategias\label{SEC:COMPARACION_ESTRATEGIAS}}

Para los grupos que obtuvieron dos \textit{playlists} diferentes, les pedíamos que valoraran por separado las dos \textit{playlists} 
y que elijieran cuál de las dos preferían. En el caso de que no obtuvieran dos \textit{playlists} diferentes, les pedíamos que 
valoraran la \textit{playlist} que les había aparecido en pantalla. De esta manera, obtenemos lo siguiente:

\begin{itemize}
    \item \textbf{Comparación de estrategias}: ninguna de las dos estrategias ha sido claramente preferida por los usuarios. Tanto en parejas, 
    como en grupos más grandes, la elección ha variado en todo momento.
    \item \textbf{Valoración de la \textit{playlist}}: tanto comparando dos, como cuando solamente se obtenía una, las puntuaciones medias
    que los usuarios han dado a las \textit{playlists} se aproximan al 6,5 sobre 10.
    \item \textbf{Obtención de dos \textit{playlists}}: para la gran mayoría de las parejas, solo se ha obtenido una \textit{playlist}, coincidiendo
    que, las que han obtenido dos \textit{playlists}, valoraban que tenían gustos homogéneos, y las que han obtenido una, que tenían gustos heterogéneos.
    Para los grupos más grandes, se han obtenido dos \textit{playlists} en todos los casos.
\end{itemize}

\subsection{Encuesta de usabilidad\label{SEC:ENCUESTA_USABILIDAD}}

Para evaluar la usabilidad de nuestra aplicación, hemos utilizado la encuesta \textit{SUS} (System Usability Scale) 
\cite{brooke1986system}. Esta encuesta son 10 preguntas sobre usabilidad, que se responden con una escala de 5 puntos, y 
nos permite obtener una puntuación de usabilidad aplicando una fórmula sobre los valores de las respuestas. Los valores 
de las respuestas que hemos utilizado han sido las medias de las respuestas de los usuarios encuestados. La fórmula para 
obtener la puntuación de usabilidad se puede ver en la Ecuación \ref{EQ:SUS} donde $r_i$ es la puntuación en la pregunta
$i$.

\begin{equation}[EQ:SUS]{Cálculo de la puntuación de usabilidad con la encuesta \textit{SUS}}
	\boxed{SUS = 2.5 \sum_{i=1}^{10}  \begin{cases}
        r_i - 1 & \text{si } i \text{ es impar} \\
        5 - r_i & \text{si } i \text{ es par}
        \end{cases}
        }
\end{equation}

Considerando la media de las puntuaciones de las respuestas de cada pregunta de la encuesta \textit{SUS}, veamos los cálculos
para obtener la puntuación de usabilidad, haciendo el ajuste a las puntuaciones de cada pregunta $p_i$, sumando todos los valores
y multiplicando por 2,5:

    \begin{align*}
        p_1 &= 3.8 - 1 = 2.8 \\
        p_2 &= 5 - 1.47 = 3.53 \\
        p_3 &= 4.53 - 1 = 3.53 \\
        p_4 &= 5 - 1.87 = 3.13 \\
        p_5 &= 4.13 - 1 = 3.13 \\
        p_6 &= 5 - 1.47 = 3.53 \\
        p_7 &= 4.67 - 1 = 3.67 \\
        p_8 &= 5 - 1.40 = 3.60 \\
        p_9 &= 4 - 1 = 3 \\
        p_{10} &= 5 - 1.53 = 3.47 \\
    \end{align*}

    \begin{align*}
        SUS &= 2.5 (2.8 + 3.53 + 3.53 + 3.13 + 3.13 + 3.53 + 3.67 + 3.60 + 3 + 3.47) \\
        SUS &= 2.5 \cdot 35.46 \\
        SUS &= 88.65
    \end{align*}

Vemos cómo la puntuación de usabilidad obtenida es de 88.65 sobre 100, lo que nos indica que la usabilidad de nuestra aplicación
es muy buena (una puntuación por encima de 70 se considera buena) tras las encuestas realizadas.

\subsection{Posible nueva funcionalidad en la aplicación\label{SEC:GESTION_USUARIOS}}

Como pregunta final de la encuesta, pedimos a los usuarios que eligieran, entre varias propuestas de nuevas funcionalidades para la 
aplicación, cuál les parecía más interesante. Las propuestas que les ofrecimos fueron las siguientes:

\begin{itemize}
    \item Poder reproducir música directamente desde la app, no tener que abrir Spotify.
    \item Añadir nuevos parámetros que influyan en la generación de las playlists, como la situación de escucha (fiesta, estudio, viaje, entrenamiento, etc.).
    \item Que haya una aplicación para móvil, no sea sólo accesible desde el navegador.
    \item Otra, tengo una sugerencia.
    \item No creo que se deba añadir nada.
\end{itemize}

La opción más votada ha sido la de añadir nuevos parámetros que influyan en la generación de las \textit{playlists}, con un 46.67\% de los votos.
La segunda opción más votada ha sido la de que haya una aplicación para móvil, con un 33.33\% de los votos. 
La opción de poder reproducir música directamente desde la app ha sido la menos votada, con un 20\% de los votos.
No ha habido otras sugerencias, y nadie ha votado que no se deba añadir nada.

\section{Resumen\label{SEC:RESUMEN_ENCUESTA}}

Después de realizar las encuestas, hemos obtenido una puntuación de usabilidad de 88.65 sobre 100, lo que es algo muy positivo para nuestra aplicación..
En cuanto a la comparación de estrategias, ninguna de las dos estrategias ha sido claramente preferida por los usuarios, y la puntuación media 
que los usuarios han dado a las \textit{playlists} ha sido aproximadamente de 6,5 sobre 10, lo cual es un resultado que muestra que, 
en general, las \textit{playlists} generadas por nuestra aplicación no han entusiasmado en exceso a los usuarios, pero tampoco les han disgustado.
También es importante destacar que, para la gran mayoría de las parejas, solo se ha obtenido una \textit{playlist}, coincidiendo con la percepción
de que tenían gustos heterogéneos (para los grupos más grandes siempre se han obtenido dos \textit{playlists}), lo que nos hace reforzar la teoría 
de que un menor número de usuarios y la baja similitud en los gustos, provoca que las estrategias de agregación no aporten diferencias en las recomendaciones.
Por último, hemos obtenido que la opción más votada para añadir nuevas funcionalidades a la aplicación ha sido la de añadir nuevos parámetros que influyan 
en la generación de las \textit{playlists}, lo cual nos deja con un reto para seguir mejorando la aplicación en el futuro.


TENGO DUDAS SOBRE LAS PREGUNTAS DE LA ENCUESTA, NO SÉ SI MOSTRARLA TODA ENTERA EN EL APÉNDICE, O MOSTRAR LAS PREGUNTAS SUELTAS EN CADA APARTADO.
LO QUE ESTA CLARO ES QUE HAY QUE MOSTRARLAS SI. 

POR OTRO LADO, EL PLANTEAMIENTO DE NO IR TRATANDO DE MANERA ESPECÍFICA CADA PREGUNTA ME GUSTA. PREFIERO IR
TRATANDO MÁS LAS CUESTIONES QUE TRATÁBAMOS DE RESPONDER CON LA ENCUESTA, Y NO IR PREGUNTA A PREGUNTA.