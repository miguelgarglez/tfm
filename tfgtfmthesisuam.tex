% arara: clean: {files: [tfgtfmthesisuam.aux, tfgtfmthesisuam.idx, tfgtfmthesisuam.ilg, tfgtfmthesisuam.ind, tfgtfmthesisuam.bbl, tfgtfmthesisuam.bcf, tfgtfmthesisuam.blg, tfgtfmthesisuam.run.xml, tfgtfmthesisuam.fdb_latexmk, tfgtfmthesisuam.fls, tfgtfmthesisuam.loe, tfgtfmthesisuam.lof, tfgtfmthesisuam.lol, tfgtfmthesisuam.lot, tfgtfmthesisuam.ltb, tfgtfmthesisuam.out, tfgtfmthesisuam.toc, tfgtfmthesisuam.upa, tfgtfmthesisuam.upb, tfgtfmthesisuam.acn, tfgtfmthesisuam.acr, tfgtfmthesisuam.alg, tfgtfmthesisuam.glg, tfgtfmthesisuam.glo, tfgtfmthesisuam.gls, tfgtfmthesisuam.glsdefs, tfgtfmthesisuam.idx,  tfgtfmthesisuam.ilg, tfgtfmthesisuam.xdy, tfgtfmthesisuam.loa, tfgtfmthesisuam.gnuploterrors , tfgtfmthesisuam.mw, tfgtfmthesisuam.fdb_latexmk ]}
% arara: pdflatex: {shell: yes}
% arara: makeglossaries
% arara: makeindex: {style: tfgtfmthesisuam.ist }
% arara: bibtex
% arara: pdflatex: {shell: yes}
% arara: pdflatex: {shell: yes}
% arara: clean: {files: [tfgtfmthesisuam.aux, tfgtfmthesisuam.idx, tfgtfmthesisuam.ilg, tfgtfmthesisuam.ind, tfgtfmthesisuam.bbl, tfgtfmthesisuam.bcf, tfgtfmthesisuam.blg, tfgtfmthesisuam.run.xml, tfgtfmthesisuam.fdb_latexmk, tfgtfmthesisuam.fls, tfgtfmthesisuam.loe, tfgtfmthesisuam.lof, tfgtfmthesisuam.lol, tfgtfmthesisuam.lot, tfgtfmthesisuam.ltb, tfgtfmthesisuam.out, tfgtfmthesisuam.toc, tfgtfmthesisuam.upa, tfgtfmthesisuam.upb, tfgtfmthesisuam.acn, tfgtfmthesisuam.acr, tfgtfmthesisuam.alg, tfgtfmthesisuam.glg, tfgtfmthesisuam.glo, tfgtfmthesisuam.gls, tfgtfmthesisuam.glsdefs, tfgtfmthesisuam.idx,  tfgtfmthesisuam.ilg, tfgtfmthesisuam.xdy, tfgtfmthesisuam.loa, tfgtfmthesisuam.gnuploterrors , tfgtfmthesisuam.mw, tfgtfmthesisuam.fdb_latexmk ]}


\documentclass[epsbased,copyright,final,printable,covers,extendedindex,firstnumbered,tfm,gnuplot]{tfgtfmthesisuam}

\usepackage{array}
\usepackage{makecell}

\advisor{Alejandro Bellogín Kouki}
\levelin{Ingeniería Informática}
\title{Desarrollo de una aplicación de recomendación de música para grupos}
\subtitle{}
\author{Miguel García González}
\privateaddress{C\textbackslash\ Ana María Nº 51}
\copyrightdate{12 de Febrero de 2024}

\dedication{A mis padres, por su apoyo incondicional.}
\famouscite{Quien recibe un beneficio nunca debe olvidarlo;\\quien lo otorga, nunca debe recordarlo. \\[0.1em] \begin{flushright}Pierre Charron\end{flushright}}
%\prefacefile{inicio/prefacio}
\ackfile{inicio/agradecimientos}
\resumenfile{inicio/resumen}
\abstractfile{inicio/abstract}

\coverdata
{
  Escuela Politécnica Superior \\
  Universidad Autónoma de Madrid \\
  C\textbackslash Francisco Tomás y Valiente nº 11
}

\bibliographyconfig{tfgtfmthesisuam}

\datadir{data}
\graphicsdir{img}
\logosdir{img}
\codesdir{codes}

\begin{document}

\lstset{
  basicstyle=\scriptsize\ttfamily,
  keywordstyle=\color{blue}, % keyword style
  stringstyle=\color{magenta}, % string literal style
  commentstyle=\color{gray}, % comment style
  showstringspaces=false,
  breaklines=true,
  postbreak=\raisebox{0ex}[0ex][0ex]{\ensuremath{\color{gray}\hookrightarrow\space}},
  frame=single,
  framerule=0.2pt,
  rulecolor=\color{gray},
  numbers=left,
  numberstyle=\scriptsize\ttfamily\color{gray},
  numbersep=8pt,
  firstnumber=1
}

\chapter{Introducción\label{CAP:INTRODUCCION}}{introduccion/introduccion}
  \section{Motivación del proyecto\label{SEC:MOTIVACION}}{introduccion/motivacion}
  \section{Propuesta y objetivos\label{SEC:OBJETIVOS}}{introduccion/objetivos}
  \section{Estructura del documento\label{SEC:ESTRUCTURA}}{introduccion/estructura}


\chapter{Estado del arte\label{CAP:ESTADODELARTE}}{estadodelarte/estadodelarte}
  \section{\textit{Framework} de desarrollo de aplicaciones: \textit{Flutter}\label{SEC:FLUTTER}}{estadodelarte/flutter}
  \section{\textit{Spotify} y su API Web\label{SEC:API_SPOTIFY}}{estadodelarte/spotify}
  \section{Recomendación\label{SEC:RECOMENDACION}}{estadodelarte/recomendacion}

\chapter{Desarrollo de nuestra aplicación\label{CAP:DESARROLLO}}{desarrollo/desarrollo}
  \section{Análisis\label{SEC:ANALISIS}}{desarrollo/analisis}
  \section{Diseño de la aplicación\label{SEC:DISENO}}{desarrollo/diseno}
  \section{Implementación de la aplicación\label{SEC:IMPLEMENTACION}}{desarrollo/implementacion}



\chapter{Pruebas y resultados\label{CAP:PRUEBAS}}{pruebas/pruebas}
  \section{Evaluaciones \textit{offline}\label{SEC:EVALUACIONES_OFFLINE}}{pruebas/offline}
  \section{Estudios con usuarios\label{SEC:ESTUDIOS_USUARIOS}}{pruebas/usuarios}

\chapter{Conclusiones y trabajo futuro\label{CAP:CONCLUSIONES}}{conclusiones/conclusiones}

% \appendix


\end{document}
