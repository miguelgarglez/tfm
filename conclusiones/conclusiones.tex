Tras la realización de este proyecto, podemos afirmar que los objetivos iniciales que planteamos en la Sección \ref{SEC:OBJETIVOS} se han cumplido.
Hemos desarrollado una aplicación web que permite a grupos de usuarios de \textit{Spotify} generar \textit{playlists} de música que combinen sus gustos musicales, consiguiendo 
una alternativa a las \textit{playlists} que la aplicación de \textit{Spotify} les pueda ofrecer. Además, con los resultados de la encuesta de usabilidad, nos llevamos un \textit{feedback}
positivo de los usuarios que nos enorgullece y nos anima a seguir trabajando en el proyecto.

Nos hemos familiarizado con el \textit{framework} de desarrollo \textit{Flutter}, y hemos desarrollado la aplicación web con él, consiguiendo asentar unos principios de desarrollo con tecnologías
\textit{frontend} que nos serán útiles en el futuro. También, hemos estudiado la API de \textit{Spotify}, y hemos desarrollado el módulo de llamadas a la API, lo que nos ha hecho coger más experiencia
y soltura en el desarrollo de aplicaciones que hagan uso de APIs de terceros que requieran autenticación.

En cuanto al otro objetivo principal del proyecto, el estudio de estrategias de agregación para la generación de \textit{playlists}, hemos obtenido resultados interesantes con los que, a pesar
de no haber llegado a conclusiones definitivas, hemos obtenido una idea de cómo podríamos seguir trabajando en el futuro. No hemos podido definir una estrategia que se alce como la mejor, pero
sí hemos podido comprobar en pruebas tanto \textit{offline} como con usuarios que cuando los grupos son de muy pocos usuarios, y los gustos son dispares, el uso de diferentes estrategias de agregación
deja de tener efecto, generando recomendaciones que son iguales.

Somos conscientes de que la aplicación desarrollada tiene mucho margen de mejora, empezando por la calidad y riqueza de las recomendaciones, y pasando por las opciones de configuración a la hora de generar
\textit{playlists}. Hemos podido ver en las encuestas que, sin duda hay posibles nuevas funcionalidades que podríamos añadir a la aplicación, destacando la posibilidad de indicar situaciones de escucha para 
las \textit{playlists} generadas. Creemos que, el \textit{endpoint} de recomendaciones de la API de \textit{Spotify} es muy rico en opciones de configuración, y que podríamos explotar más estas opciones para
mejorar la calidad de las recomendaciones y, llevar a la realidad la nueva funcionalidad.

Otra cuestión que nos surge a raíz de las valoraciones en las encuestas es que, nos gustaría probar otra estrategia muy simple, en la cual, simplemente pusieramos en una \textit{playlist} las canciones más escuchadas
de cada usuario del grupo. Creemos que, en el contexto de nuestras pruebas, habría desencadenado resultados muy positivos, debido a que las personas identificarían a simple vista canciones que conocen y que saben que les gustan, que 
eclipsarían psoiblemente a las canciones que no conocen. También, otra variables a tener en cuenta para más experimentos, serían las ventanas temporales utilizadas para obtener los artistas y canciones más escuchados de los usuarios, 
ya que podría tener cierta relevancia.

La limitación que pone \textit{Spotify} sobre los usuarios que pueden utilizar la aplicación (tenemos que registrar los correos de quien la va a utilizar previamente), es un problema cuya única solución es hacer una solicitud 
reglamentaria a \textit{Spotify} para que cambie el plan de la aplicación a uno que permita a cualquier usuario de \textit{Spotify} utilizarla. Tenemos como trabajo futuro mejorar y adaptar la aplicación para que esta cumpla con 
los requisitos de \textit{Spotify}, y así poder llegar a muchos más usuarios, algo que sería muy beneficioso para los experimentos.

En cuanto a la aplicación en sí, nos gustaría trabajar en mejorar la interfaz de usuario, haciéndola más atractiva y usable, por ejemplo, añadiendo algunas animaciones de progreso y carga, desarrollar un modo oscuro y uno claro,
o la opción de tener más de un idioma. Además, trabajar en el rendimiento y fluidez de la aplicación es algo que nos gustaría hacer, ya que, aunque no hemos tenido problemas notorios de rendimiento, creemos que siempre se puede mejorar.

En resumen, estamos satisfechos con el trabajo realizado, y creemos que hemos cumplido con los objetivos que nos planteamos. Hemos aprendido mucho, y nos llevamos un proyecto que nos ha gustado mucho desarrollar, 
y que nos deja con ganas de seguir trabajando en él. Por último, dejamos el enlace al repositorio de \textit{GitHub} del proyecto con el código: \url{https://github.com/miguelgarglez/combined_playlist_maker}.