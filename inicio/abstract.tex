That technology inundates our lives after progressing rapidly is an obvious fact.
Music is no exception, and the emergence of streaming platforms like Spotify has changed the way we listen to music.
Obtaining listening data from users allows their music preferences to be profiled, shifting the weight of choice from the user, 
who previously had to manually search for music they would have previously discovered they liked, to the system, which recommends 
music that best matches the user's registered preferences. Currently, Spotify translates this recommendation into automatically 
generated and updated playlists. Users have become completely passive in these recommendations, and while their appeal, usefulness, 
and effectiveness are not questioned, this work aims to explore an intermediate alternative in which both the user and the system 
play a role in generating recommendations.

To this end, this document presents the development of a web application that allows Spotify users to generate playlists based on the 
preferences of a group of Spotify users. The Spotify API will be used to obtain user data, most listened songs, and recommendations. 
The Flutter framework (developed by Google), which allows for cross-platform development with a single codebase, will be used for development. 
In this case, the web version of the application will be used. The code for the developed application can be found in the GitHub repository at 
\url{https://github.com/miguelgarglez/combined_playlist_maker}.

User tests will be conducted to evaluate the application and different ways of generating playlists using aggregation strategies. It will be 
observed that in many contexts, it is difficult to generate playlists with different aggregation strategies, and it is even more challenging to 
satisfy users from a group with diverse musical tastes. Subsequently, the obtained results will be analyzed to gather feedback, draw conclusions, 
and identify possible future work related to the conducted research.

\keywords{Flutter, Spotify, recommendation, group recommendation, aggregation strategies}
