That technology inundates our lives after having progressed rapidly is an obvious fact.
Music is no exception, and the emergence of streaming platforms like Spotify has changed the way we listen to music.
Obtaining listening data from users allows their music preferences to be profiled, shifting the weight of choice 
from the user, who previously had to manually search for music they would have previously 
discovered they liked, to the system, which recommends music 
that best matches the user's registered preferences. Currently, Spotify translates this recommendation into automatically generated and updated playlists.
Users have become completely passive in these recommendations, and
although their appeal, usefulness, and effectiveness are not questioned, this work aims to explore an intermediate alternative; one in which the user and the system
play a role in generating recommendations.

To this end, this document presents the development of a web application that allows Spotify users
to generate playlists based on the preferences of a group of Spotify users. The Spotify API will be used
to obtain user data, most listened songs, and recommendations. The development framework used will be Flutter (developed by Google), which allows for cross-platform application development with a single codebase. In this case, the web version of the application will be used.

Tests will be conducted with users to evaluate the application and different ways of generating playlists. Afterwards, the obtained results will be analyzed to gather feedback, draw conclusions, and identify possible future work based on the work done.

\keywords{Flutter, Spotify, recommendation, group recommendation, aggregation strategies}
