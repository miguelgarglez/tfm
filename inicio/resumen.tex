Que la tecnología inunda nuestras vidas tras haber progresado de manera vertiginosa es una obviedad.
La música no es una excepción, y la aparición de plataformas de \textit{streaming} como \textit{Spotify} ha cambiado la forma en la que escuchamos música.
La obtención de datos de escucha de los usuarios permite que se puedan perfilar sus gustos de música, dejando de estar todo el peso 
de la elección en el usuario, que antes tendría que buscar manualmente la música que él previamente tendría 
que haber descubierto que le gustase; a estar una gran parte del peso en el sistema, que recomendará la música 
que más se ajuste a los gustos registrados del usuario. A día de hoy \textit{Spotify} traduce esta recomendación en listas de reproducción (\textit{playlists})
generadas automáticamente y que se actualizan automáticamente. Los usuarios han pasado a tener un papel totalmente pasivo en estas recomendaciones, y
aunque no se quiere poner en duda su atractivo o utilidad y eficacia, en este trabajo se quiere explorar una alternativa intermedia; en la cual, usuario y sistema
tengan un papel en la generación de recomendaciones.

Para ello, se expondrá en este documento el desarrollo de una aplicación web que permite a los usuarios de \textit{Spotify}
generar listas de reproducción (\textit{playlists}) de música que se basen en los gustos de un grupo de usuarios de \textit{Spotify}. Se utilizará la API de \textit{Spotify}
para obtener datos de usuarios, canciones más escuchadas, y recomendaciones. Como \textit{framework} de desarrollo se utilizará \textit{Flutter} (desarrollado por \textit{Google}), que permite desarrollar aplicaciones
multiplataforma, con una misma base de código. En este caso se hará uso de la versión web de la aplicación.

Se llevarán a cabo pruebas con usuarios para evaluar la aplicación y distintas formas de generar \textit{playlists}. Tras ello, se analizarán los resultados obtenidos para obtener 
\textit{feedback}, conclusiones y posible trabajo futuro sobre el trabajo realizado.

\palabrasclave{Flutter, \textit{Spotify}, recomendación, recomendación de grupos, estrategias de agregación}
