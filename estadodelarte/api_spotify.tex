En esta sección realizaremos una visión general de la API de \textit{Spotify} \cite{spotify_api}, e iremos pasando por
los \textit{endpoints} más destacados, que serán útiles para el desarrollo de la aplicación. No obstante, 
en el capítulo \ref{CAP:DESARROLLO} se ahondará más en el uso de cada uno de ellos.

\subsection{Visión general de la API\label{subsec:vision_general_api}}

La API de \textit{Spotify} \cite{spotify_api} es una API \textit{REST}, que utiliza el formato \textit{JSON} para el intercambio 
de datos. Es importante destacar que para hacer uso de esta API, es necesario darse de alta como desarrollador, asumiendo que se 
tiene una cuenta de \textit{Spotify}. Una vez hecho esto, se podrán dar de alta aplicaciones en la cuenta de desarrollador desde el
\textit{Dashboard}, y tener así los datos y credenciales necesarios para hacer uso de la API. 
En el \textit{Dashboard} se pueden ver las aplicaciones dadas de alta, así como crear nuevas. Además, se puede entrar en cada
aplicación para ver datos de utilización de \textit{endpoints} y editar ajustes sobre ella, como la dirección del sitio web,
las posibles \textit{redirect URIs} para el flujo de autenticación, etc.
Obsérvese la figura \ref{FIG:SPOTIFY_DEV}
a modo de orientación.

\begin{figure}[Visión general \textit{Spotify for Developers}]{FIG:SPOTIFY_DEV}
    {Visión general \textit{Spotify for Developers} \\
    {\scriptsize Imágenes extraídas de \href{https://developer.spotify.com/dashboard/}{developer.spotify.com}}}
    \subfigure[]{\textit{Dashboard}}{\image{7cm}{}{propias/dashboard.png}} \quad
    \subfigure[]{Información aplicación}{\image{7cm}{}{propias/app_basic_info.png}} \quad
    \subfigure[]{Editar ajustes aplicación}{\image{7cm}{}{propias/app_settings.png}}
\end{figure}

Para hacerse una idea inicial de qué tipos de \textit{endpoints} nos harán falta, se puede hacer una primera aproximación
a la aplicación que se quiere desarrollar. Vamos a suponer que un grupo de usuarios de \textit{Spotify} está 
reunido e interesado en crear una playlist común que combine sus gustos musicales para escuchar en ese momento
(o en un futuro). Teniendo en cuenta esta situación, y el resultado que se quiere obtener, podemos
listar qué tipos de \textit{endpoints} o interacciones con la API nos harán falta:

\begin{itemize}
  \item \textbf{Datos de cuenta de los usuarios}, es decir, necesitaremos que los usuarios hagan 
  \textit{login} con sus cuentas de \textit{Spotify} 
  \item \textbf{Canciones más escuchadas} porque es necesario saber sobre los gustos de los usuarios y,
  obtener las canciones más escuchadas de cada uno de los usuarios que forman el grupo, parece una buena forma de hacerlo.
  \item \textbf{Recomendaciones}, porque así, podremos aprovecharnos de las estrategias de recomendación 
  individuales de \textit{Spotify}, y obtener recomendaciones de canciones que puedan gustar a los usuarios 
  que forman el grupo, para así generar la playlist común.
  \item \textbf{Crear playlist}, porque así los usuarios podrán guardar en sus cuentas de \textit{Spotify} la
  playlist que se haya generado.
\end{itemize}

\subsection{Datos de cuenta de los usuarios\label{subsec:datos_cuenta_usuarios}}

Para que los usuarios puedan autenticarse aplicaciones con sus cuentas de \textit{Spotify}, y así poder crear playlists 
combinadas con los usuarios de su grupo, se llevará a cabo el flujo de autenticación PKCE 
(\textit{Proof Key for Code Exchange}) definido en \cite{spotify_pkce}, el cual es el recomendado por 
\textit{Spotify} para aplicaciones móviles u aplicaciones web de una sola página, donde la clave secreta
del cliente no puede ser protegida de manera segura. Se puede ver un gráfico del flujo en la figura \ref{FIG:SPOTIFY_PKCE_FLOW}.

\begin{figure}[Flujo de autenticación PKCE]{FIG:SPOTIFY_PKCE_FLOW}
    {Flujo de autenticación PKCE}
          \image{14cm}{}{propias/flujo_pkce.png}
\end{figure}

Para llevar a cabo esta autenticación, la app tendrá que lanzar la URL de autenticación de \textit{Spotify}
("https://accounts.spotify.com/authorize"). Para más detalles sobre cómo se lleva a cabo este flujo en nuestra app, se tratará en 
el capítulo \ref{CAP:DESARROLLO}.

\subsection{Canciones más escuchadas\label{subsec:canciones_mas_escuchadas}}

\subsection{Recomendaciones\label{subsec:recomendaciones}}

\subsection{Crear playlist\label{subsec:crear_playlist}}



