Como introducción a esta sección, un fragmento del prefacio del libro \textit{Recommender Systems Handbook} \cite{recommender_handbook}
traducido por nosotros al español aporta una visión bastante actual de qué debe venirnos a la cabeza cuando hablamos de sistemas de recomendación:

\textit{``Los sistemas de recomendación son herramientas y técnicas de software que ofrecen sugerencias de artículos 
útiles para un usuario. Las sugerencias de un sistema de recomendación están pensadas para ayudar al usuario a tomar 
decisiones como qué comprar, qué música escuchar o qué noticias leer. Los sistemas de recomendación son un valioso medio
 para que los usuarios de Internet hagan frente a la sobrecarga de información y les ayuden a elegir mejor. En la 
 actualidad son una de las aplicaciones más populares de la inteligencia artificial, ya que ayudan a descubrir información
 en la Red. Se han propuesto varias técnicas de generación de recomendaciones y, en las dos últimas décadas, muchas de
 ellas se han implantado con éxito en entornos comerciales. Hoy en día, todos los grandes actores de Internet adoptan técnicas de
 recomendación. El desarrollo de sistemas de recomendación es un esfuerzo multidisciplinar en el que participan expertos 
 de diversos campos, como la inteligencia artificial, la interacción persona-ordenador, la minería de datos, la estadística, 
 los sistemas de apoyo a la decisión, el marketing y el comportamiento del consumidor.''}

El estudio sobre sistemas de recomendación es un campo bastante más nuevo que otras técnicas y herramientas clásicas de los sistemas de información, 
como las bases de datos, o los motores de búsqueda. A pesar de ser un área de estudio independiente relativamente joven, el interés en estos sistemas
ha crecido de manera muy significativa. Simplemente, un claro signo de ello es cómo se hace uso de los sistemas de recomendación en la actualidad,
como por ejemplo en las plataformas de \textit{streaming} de música o vídeo (como \textit{Spotify} o \textit{Netflix}), o en las plataformas de
comercio electrónico (como \textit{Amazon} o \textit{eBay}). Reflexionemos sobre el motivo de su extendido uso. 

\textbf{Partes interesadas en un sistema de recomendación}

Según \cite{recommenderintro}, en un sistema de recomendación influyen tres partes interesadas: los usuarios (o consumidores), los proveedores (o suministradores)
y los propietarios del sistema. Los usuarios son los que reciben las recomendaciones (basándonos en este trabajo, digamos que los usuarios de \textit{Spotify}), 
los proveedores son los que ofrecen los productos o servicios que se recomiendan (digamos que los artistas que publican su música en \textit{Spotify}), y
el propietario del sistema (en este caso, \textit{Spotify}) que ofrece la plataforma en la que los usuarios son acercados a los proveedores (artistas). Sabiendo
esto, podemos ver que las tres partes se ven beneficiadas ante un escenario ideal: los usuarios reciben recomendaciones de productos o servicios que les
interesan (descubren nuevas canciones, géneros y artistas que les gustan), los proveedores ven aumentada su visibilidad (los artistas pueden volverse más
conocidos) y, por tanto, sus ventas, y el propietario del sistema ve aumentado su número de usuarios y, por tanto, sus ingresos (más suscripciones de \textit{Spotify}).
Este beneficio claro para las tres partes diríamos que es claramente uno de los mayores motivos por los que los sistemas de recomendación son tan 
populares y utilizados en la actualidad.

\textbf{Fuentes de datos y conocimiento en un sistema de recomendación}

Por otro lado, no podemos dejar de lado la suma importancia que tienen los datos de los que hace uso un sistema de recomendación. En \cite{recommenderintro}, 
se destacan tres elementos como las fuentes de conocimiento y datos de las que un sistema de recomendación se nutre: los usuarios, los productos (o ítems)
y las interacciones entre los dos anteriores. Los \textbf{ítems}, que son los objetos de las recomendaciones, deben contar con una descripción detallada y exhaustiva, 
abarcando desde características generales hasta metadatos específicos. Algoritmos especializados extraen y analizan estas características, los cuales son
especialmente necesarios y útiles para ítems complejos como imágenes o textos, utilizando técnicas avanzadas de procesamiento de imágenes 
y lenguaje natural.
En cuanto a los \textbf{usuarios}, cada usuario se modela de manera única, integrando una diversidad de factores como intereses, edad y género, entre otros. 
Pero más allá de las preferencias que se puedan mostrar explícitamente (como que un usuario indique sus géneros de música favoritos de manera activa),
el modelo de un usuario también se enriquece con información contextual, permitiendo así que el sistema de recomendación ajuste sus sugerencias a las
circunstancias específicas y momentáneas del usuario (como un descubrimiento de un nuevo artista o género que está escuchando recientemente).
Por último, las \textbf{interacciones} entre usuarios e ítems son la fuente de datos más importante para un sistema de recomendación. Desde calificaciones
explícitas hasta comportamientos implícitos de navegación, cada acción del usuario aporta información valiosa, contribuyendo a la comprensión profunda de 
sus preferencias y comportamientos.

\subsection{Métodos y técnicas de recomendación\label{SEC:METODOS_RECOMENDACION}}


    \subsubsection{Filtrado basado en contenido\label{SEC:RECOMENDADORES_CONTENIDO}}
    \subsubsection{Filtrado colaborativo\label{SEC:RECOMENDADORES_COLABORATIVOS}}
    \subsubsection{Sistemas híbridos\label{SEC:RECOMENDADORES_HIBRIDOS}}

\subsection{Tipos de sistemas de recomendación\label{SEC:TIPOS_RECOMENDADORES}}
    \subsubsection{Recomendación de música\label{SEC:RECOMENDACION_MUSICA}}
    \subsubsection{Recomendación a grupos\label{SEC:RECOMENDACION_GRUPOS}}

\subsection{Estrategias de agregación\label{SEC:AGREGACION}}




