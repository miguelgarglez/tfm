Como introducción a esta sección, un fragmento del prefacio del libro \textit{Recommender Systems Handbook} \cite{recommender_handbook}
traducido por nosotros al español aporta una visión bastante actual de qué debe venirnos a la cabeza cuando hablamos de sistemas de recomendación:

\textit{``Los sistemas de recomendación son herramientas y técnicas de software que ofrecen sugerencias de artículos 
útiles para un usuario. Las sugerencias de un sistema de recomendación están pensadas para ayudar al usuario a tomar 
decisiones como qué comprar, qué música escuchar o qué noticias leer. Los sistemas de recomendación son un valioso medio
 para que los usuarios de Internet hagan frente a la sobrecarga de información y les ayuden a elegir mejor. En la 
 actualidad son una de las aplicaciones más populares de la inteligencia artificial, ya que ayudan a descubrir información
 en la Red. Se han propuesto varias técnicas de generación de recomendaciones y, en las dos últimas décadas, muchas de
 ellas se han implantado con éxito en entornos comerciales. Hoy en día, todos los grandes actores de Internet adoptan técnicas de
 recomendación. El desarrollo de sistemas de recomendación es un esfuerzo multidisciplinar en el que participan expertos 
 de diversos campos, como la inteligencia artificial, la interacción persona-ordenador, la minería de datos, la estadística, 
 los sistemas de apoyo a la decisión, el marketing y el comportamiento del consumidor.''}

El estudio sobre sistemas de recomendación es un campo bastante más nuevo que otras técnicas y herramientas clásicas de los sistemas de información, 
como las bases de datos, o los motores de búsqueda. A pesar de ser un área de estudio independiente relativamente joven, el interés en estos sistemas
ha crecido de manera muy significativa. Un claro signo de ello es cómo se hace uso de los sistemas de recomendación en la actualidad,
como por ejemplo en las plataformas de \textit{streaming} de música o vídeo (como \textit{Spotify} o \textit{Netflix}), o en las plataformas de
comercio electrónico (como \textit{Amazon} o \textit{eBay}). Reflexionemos sobre el motivo de su extendido uso. 

\textbf{Partes interesadas en un sistema de recomendación}

Según \cite{recommenderintro}, en un sistema de recomendación influyen tres partes interesadas: los usuarios (o consumidores), los proveedores (o suministradores)
y los propietarios del sistema. Los usuarios son los que reciben las recomendaciones (basándonos en este trabajo, digamos que los usuarios de \textit{Spotify}), 
los proveedores son los que ofrecen los productos o servicios que se recomiendan (digamos que los artistas que publican su música en \textit{Spotify}), y
el propietario del sistema (en este caso, \textit{Spotify}) que ofrece la plataforma en la que los usuarios son expuestos a los proveedores (artistas). Sabiendo
esto, podemos ver que las tres partes se ven beneficiadas ante un escenario ideal: los usuarios reciben recomendaciones de productos o servicios que les
interesan (descubren nuevas canciones, géneros y artistas que les gustan), los proveedores ven aumentada su visibilidad (los artistas pueden volverse más
conocidos) y, por tanto, sus ventas, y el propietario del sistema ve aumentado su número de usuarios y, por tanto, sus ingresos (más suscripciones de \textit{Spotify}).
Este beneficio claro para las tres partes diríamos que es uno de los mayores motivos por los que los sistemas de recomendación son tan 
populares y utilizados en la actualidad.

\textbf{Fuentes de datos y conocimiento en un sistema de recomendación}

Por otro lado, no podemos dejar de lado la suma importancia que tienen los datos de los que hace uso un sistema de recomendación. En \cite{recommenderintro}, 
se destacan tres elementos como las fuentes de conocimiento y datos de las que un sistema de recomendación se nutre: los usuarios, los productos (o ítems)
y las interacciones entre los dos anteriores. Los \textbf{ítems}, que son los objetos de las recomendaciones, deben contar con una descripción detallada y exhaustiva, 
abarcando desde características generales hasta metadatos específicos. Algoritmos especializados extraen y analizan estas características, los cuales son
especialmente necesarios y útiles para ítems complejos como imágenes o textos, utilizando técnicas avanzadas de procesamiento de imágenes 
y lenguaje natural.
En cuanto a los \textbf{usuarios}, cada usuario se modela de manera única, integrando una diversidad de factores como intereses, edad y género, entre otros. 
Pero más allá de las preferencias que se puedan mostrar explícitamente (como que un usuario indique sus géneros de música favoritos de manera activa),
el modelo de un usuario también se enriquece con información contextual, permitiendo así que el sistema de recomendación ajuste sus sugerencias a las
circunstancias específicas y momentáneas del usuario (como un descubrimiento de un nuevo artista o género que está escuchando recientemente).
Por último, las \textbf{interacciones} entre usuarios e ítems son la fuente de datos más importante para un sistema de recomendación. Desde calificaciones
explícitas hasta comportamientos implícitos de navegación, cada acción del usuario aporta información valiosa, contribuyendo a la comprensión profunda de 
sus preferencias y comportamientos.

\subsection{Métodos y técnicas de recomendación\label{SEC:METODOS_RECOMENDACION}}

Para que un sistema de recomendación haga bien su trabajo, que es sugerir ítems que puedan interesar al usuario, tiene que ser capaz de adivinar o predecir
 cómo de útiles o atractivos serán ciertos ítems para cada persona. Esto se hace evaluando y comparando lo que se conoce sobre los ítems y 
 los usuarios. A veces, por ejemplo, cuando no hay mucha información específica sobre los gustos de un usuario, el sistema puede optar por recomendar lo que es popular
  o lo que ha gustado a muchos, pensando que hay más posibilidades de que también le guste a ese usuario.

En términos técnicos, se habla de calcular la 'utilidad' (también lo llamaremos \textit{rating} en este trabajo) de un ítem para un usuario, que no es más
 que tratar de prever cuánto le va a gustar. No todos los sistemas hacen este cálculo de manera completa y obtienen un \textit{rating} como tal; algunos 
 se basan en métodos más generales o en suposiciones para decidir qué recomendar. Además, hay que considerar que lo que a una persona le puede parecer
 atractivo o interesante puede depender de muchas cosas, como la actividad que puede estar haciendo, su compañía (si la tiene) o el lugar, así que las recomendaciones 
 también deberían tener en cuenta esos detalles idealmente, para que fueran realmente acertadas.

 Por supuesto, existen diversas formas en las cuales un sistema de recomendación puede calcular estas predicciones de utilidad o atractivo para los usuarios.
 Trataremos tres tipos principales: basado en contenido, el filtrado colaborativo y los sistemas híbridos. 


 \subsubsection{Basado en contenido\label{SEC:RECOMENDADORES_CONTENIDO}}
 La recomendación basada en contenido se centra en las características de los ítems para hacer recomendar a los usuarios. Funciona analizando los 
 detalles de los ítems que a un usuario le han gustado en el pasado y buscando otros ítems que sean similares en cuanto a esas características. Por 
 ejemplo, si a alguien le gustan las películas de un género en particular o de un director específico, el sistema recomendará otras películas que 
 coincidan con esas preferencias. De manera muy sencilla, la recomendación basada en contenido mira lo que te ha gustado antes y te sugiere cosas parecidas. 
 Imaginemos que nos encantan ciertas canciones de pop español o las canciones de un artista en concreto. El sistema observa esto y luego te muestra más películas del mismo tipo.

 Este método utiliza algoritmos para procesar y entender las características de los ítems, como las palabras en una descripción, los elementos 
 visuales en una imagen, o el análisis de audio de una canción. Con esta información, el sistema crea un perfil de lo que le gusta al usuario basado en 
 lo que ha elegido anteriormente y luego busca y recomienda nuevos ítems que encajen con ese perfil. Así, la recomendación basada en contenido permite que las sugerencias sean 
 personalizadas y relevantes para los intereses específicos de cada usuario.


\subsubsection{Filtrado colaborativo\label{SEC:RECOMENDADORES_COLABORATIVOS}}

El filtrado colaborativo es el enfoque original y más simple de los sistemas de recomendación. Predice la utilidad de un ítem para un usuario basándose en
cómo de atractivo o útil les ha parecido ese ítem a otros usuarios. El fundamento detrás de esto es que, si dos usuarios han tenido intereses parecidos en el
pasado, desde luego es probable que tengan intereses parecidos en el futuro. Imaginemos que dos usuarios han escuchado las mismas canciones de un artista
en el pasado, o de un género en particular. Si uno empieza a escuchar una nueva canción de manera recurrente (le gusta), es probable que al otro usuario
también le vaya a gustar esta canción. Podemos imaginarlo como si tenemos un amigo con el que compartimos gustos musicales; lógicamente, tendremos en cuenta 
sus recomendaciones, ya que sabemos que tenemos gustos parecidos.

Debe destacarse que el filtrado colaborativo no necesita información detallada sobre los ítems, como la recomendación basada en contenido. Existen dos tipos
de filtrado colaborativo: basado en usuarios y basado en ítems. El basado en usuarios compara a los usuarios entre sí y recomienda ítems que les han parecido
atractivos a otros usuarios identificados como similares al usuario al que se le recomienda; mientras que, el basado en ítems compara
ítems entre sí. Debemos dejar clara la diferencia entre el filtrado colaborativo basado en ítems, y la recomendación basada en contenido.

En la recomendación basada en contenido (que hemos comentado en la Sección \ref{SEC:RECOMENDADORES_CONTENIDO}), se modelan con procesos y algoritmos especiales las características de los ítems
 que consume un usuario para, tras ello, recomendarle más ítems que se asemejen a las características de ese modelo producido. En cambio, en el filtrado
  colaborativo enfocado en ítems, la diferencia fundamental se encuentra en cuándo se considera a dos ítems similares entre sí (obviando la diferencia de 
  que aquí se comparan ítems directamente, y en la basada en contenido se comparan ítems con un perfil o modelo de características extraído de haber 
  procesado los ítems consumidos previamente). Y es que, si un conjunto de usuarios ha calificado o interactuado de manera similar con ciertos ítems, 
  entonces esos ítems se consideran similares entre sí. No se necesitan detalles sobre las características de los ítems, como género de una canción, 
  artista o análisis de audio. La similitud entre los ítems se calcula en función de las calificaciones o interacciones de los usuarios, no en función 
  de sus características intrínsecas. Aún así, estas caracteristicas intrínsecas podrían tenerse en cuenta para, posiblemente, mejorar la calidad de las
  recomendaciones, tratando de coger lo mejor de los dos tipos de sistemas. Veamos a continuación los Sistemas Híbridos.


\subsubsection{Sistemas híbridos\label{SEC:RECOMENDADORES_HIBRIDOS}}

Estos sistemas nacen de la necesidad de cubrir los inconvenientes con los que nos podemos encontrar con los anteriores métodos. No entraremos en más detalles,
pero por ejemplo, mientras que los métodos de filtrado colaborativo tienen dificultades para recomendar nuevos ítems sin calificaciones, los enfoques 
basados en contenido pueden hacer predicciones para estos ítems usando sus descripciones.

Además, es importante considerar el contexto del usuario para personalizar más las recomendaciones. Por ejemplo, las sugerencias para vacaciones en 
invierno deberían ser distintas a las de verano, o las recomendaciones de canciones deberían ser diferentes para una sesión de ejercicio, de estudio, o 
una fiesta. Los sistemas de recomendación sensibles al contexto (CARS) \cite{contextrecommender} utilizan esta información para afinar sus sugerencias.




\subsection{Tipos de sistemas de recomendación\label{SEC:TIPOS_RECOMENDADORES}}

Tras haber visto como base los métodos más comunes de recomendación, vamos a pasar a hablar de los dos ámbitos o tipos de recomendación que más nos
interesan en este trabajo por razones obvias si queremos desarrollar una aplicación como la descrita en la introducción: 
la recomendación de música y la recomendación a grupos.
    
\subsubsection{Recomendación de música\label{SEC:RECOMENDACION_MUSICA}}

Aunque en \cite{musicrecommender} se habla de manera bastante extensa sobre la recomendación en este ámbito, trataremos de hacer un resumen con los puntos que 
más creemos que nos interesan para este trabajo. Está claro que el consumo de música ha cambiado mucho en los últimos años, llegando a la situación actual,
donde la música se consume de manera digital, generalmente en \textit{streaming}. Esto ha llevado a que estas plataformas que dan este tipo de servicio 
lleguen a ofrecer unas cantidades de música (o contenido, si incluimos los \textit{podcasts}) tan inmensas y tan variadas que son inabarcables para nadie. 
Y aquí es donde entra en juego la recomendación de música, que se convierte en una herramienta muy útil para los usuarios, al ayudarles a navegar por el inmenso
catálogo de música que ofrecen estas plataformas.

Por otro lado, como se expone en \cite{musicrecommender}, hay unos cuantos factores que hacen que los sistemas de recomendación de música se diferencien considerablemente 
de otros tipos de sistemas de recomendación:

\begin{itemize}
    \item \textbf{Duración de Consumo:} La música se consume más rápidamente que otros medios (como películas o libros), lo que significa que los usuarios
     pueden formar opiniones sobre los ítems musicales en menos tiempo. Esto puede hacer que los ítems musicales se consideren más 'desechables'.
    
    \item \textbf{Tamaño del Catálogo:} Los catálogos de música son considerablemente más grandes que los de películas o series. Por lo tanto, la 
    escalabilidad de los algoritmos de recomendación es una consideración más crítica en el dominio de la música.
    
    \item \textbf{Consumo Repetido:} Los ítems musicales a menudo se consumen repetidamente, incluso varias veces seguidas, a diferencia de otros medios
     como películas o libros.
    
    \item \textbf{Consumo Secuencial:} Las canciones suelen consumirse en secuencia (como en \textit{playlists}), lo que da importancia a problemas 
    de recomendación que tienen en cuenta la secuencia, como la continuación automática de listas de reproducción o la recomendación de la próxima canción.
    
    \item \textbf{Consumo Pasivo:} La música a menudo se consume de manera pasiva, lo que puede afectar la calidad de los indicios de preferencia,
    especialmente obteniéndolos de manera implícita (como el tiempo de escucha).
    
    \item \textbf{Importancia del Contenido:} En la recomendación musical, los enfoques basados en contenido son más relevantes que en otros dominios 
    debido a la influencia de campos como el procesamiento y análisis de audio, y la recuperación de información musical. Estos enfoques intentan extraer 
    información semántica de la música y usarla para hacer recomendaciones, a menudo debido a la escasez de datos de calificación explícitos en este 
    dominio.
\end{itemize}


\subsubsection{Recomendación a grupos\label{SEC:RECOMENDACION_GRUPOS}}

También se habla de forma extensa sobre este ámbito de los sistemas de recomendación en \cite{grouprecommender}, pero trataremos de destacar los puntos
que más nos pueden interesar para el trabajo y el desarrollo de nuestra idea de aplicación presentada en la introducción. A continuación, exponemos
una serie de características que permiten clasificar los sistemas de recomendación a grupos:

\begin{itemize}
    \item \textbf{Conocimiento de Preferencias:} Las preferencias individuales pueden ser conocidas de antemano o desarrollarse con el tiempo a través 
    de interacciones de grupo u obtención de valoraciones.
    \item \textbf{Experiencia de los Ítems:} Los ítems pueden ser experimentados directamente por el grupo (como música o anuncios) o presentados como 
    opciones en una lista de recomendaciones.
    \item \textbf{Interactividad del Grupo:} El grupo puede ser pasivo, sin influir en la agregación de preferencias, o activo, participando en la 
    negociación o influenciando las calificaciones.
    \item \textbf{Negociación vs. Recomendaciones Directas:} Algunos sistemas permiten negociación y discusión sobre las recomendaciones, mientras que 
    otros entregan recomendaciones de forma directa sin interacción.
    \item \textbf{Recomendación de Ítem Único vs. Secuencia:} Algunos sistemas recomiendan un único ítem, mientras que otros sugieren secuencias de ítems,
     como listas de reproducción o rutas turísticas.
    \item \textbf{Estimación de Preferencias:} Las preferencias individuales pueden ser estimadas usando filtrado basado en contenido o colaborativo.
    \item \textbf{Agregación de Perfiles o Recomendaciones:} Los sistemas pueden agregar perfiles de usuarios, que expresan preferencias numéricas por ítem, 
    o directamente las recomendaciones, ya sea como listas clasificadas o como estimaciones numéricas.
\end{itemize}

Un tema importante en la recomendación dirigida a grupos es la agregación de preferencias, y en nuestro caso la agregación de las recomendaciones 
individuales, en una sola para todo el grupo. Y este tema lo trataremos en la siguiente sección, ya que, es el factor sobre el que han recaído 
algunas pruebas, investigación y experimentos con usuarios a la hora de probar distintas generaciones de \textit{playlists}.

\subsection{Estrategias de agregación\label{SEC:AGREGACION}}

En la recomendación dirigida a grupos, el principal reto a resolver es cómo conseguir tratar a un grupo de usuarios como una única entidad, basándonos
en los gustos individuales de cada uno de los miembros de este. En \cite{grouprecommender} se habla de varias estrategias de agregación de preferencias,
estas sirven de la misma forma para agregar perfiles, y agregar recomendaciones (lo que haremos en nuestro sistema de recomendación). 
Veamos algunas de las estrategias de agregación más comunes, las nombraremos en inglés, para minimizar confusiones entre 
la bibliografía, esta sección y el desarrollo:

\begin{itemize}
    \item \textbf{\textit{Average}:} Promedia las calificaciones de los usuarios.
    \item \textbf{\textit{Multiplicative}:} Multiplica las calificaciones de los usuarios. 
    \item \textbf{\textit{Borda}:} Asigna puntos a cada ítem en función de su clasificación por cada usuario, y luego suma los puntos 
    para cada ítem.
    \item \textbf{\textit{Least Misery}:} Asigna a cada ítem el valor mínimo de las calificaciones de los usuarios.
    \item \textbf{\textit{Most Pleasure}:} Asigna a cada ítem el valor máximo de las calificaciones de los usuarios.
    \item \textbf{\textit{Average without Misery}:} Asigna a cada ítem el valor promedio de las calificaciones de los usuarios, pero
    excluye las calificaciones más bajas (se establece un umbral para decidir cuándo es demasiado baja una puntuación o \textit{rating}).
    \item \textbf{\textit{Fairness}:} Los ítems se ordenan como si cada usuario eligiera un ítem por turnos. Así, si hay tres usuarios, el
    los primeros tres ítems de la lista serán los que más gusten a cada usuario, y así sucesivamente.
\end{itemize}


\subsection{Evaluación de un sistema de recomendación a grupos\label{SEC:EVALUACION}}

Cómo evaluar un sistema de recomendación sigue siendo un tema de investigación muy importante a día de hoy, ya que, a pesar de que existen métricas o 
pruebas con usuarios, no hay un método de evaluación que sea el más adecuado para todos los sistemas de recomendación. En \cite{grouprecommender} se habla
sobre la evaluación de sistemas de recomendación dirigidos a grupos, y se destacan dos tipos: estudios con usuarios, y evaluaciones \textit{offline}.

\subsubsection{Estudios con usuarios}

Se lleva a cabo una investigación con usuarios cuando queremos evaluar cómo rinde un sistema basándonos en aspectos como lo que fácil es de usar o cuánto 
les gusta a los usuarios (por ejemplo, si están satisfechos con él o cómo de buenas son las recomendaciones que reciben). Para recoger esta información, a 
veces hablamos directamente con las personas involucradas o utilizamos plataformas online que permiten la participación masiva. Aunque estos estudios son 
esenciales para entender si la gente realmente valora y acepta el sistema, no debemos depender únicamente de ellos para medir la efectividad del sistema, 
ya que su alcance es limitado. Sin embargo, entender la percepción y la experiencia del usuario es fundamental para confirmar si nuestra solución realmente 
cumple con las necesidades de quienes la utilizan.

\subsubsection{Evaluaciones \textit{offline}}

En la investigación sobre sistemas de recomendación para grupos, al igual que en los individuales, se emplean evaluaciones \textit{offline}. No obstante, 
la falta de datos reales sobre las preferencias de grupos complica estos estudios. Una solución común es el uso de grupos sintéticos, creados artificialmente
 a partir de datos estándar. El problema principal es cómo validar efectivamente las recomendaciones para estos grupos sintéticos, dado que las preferencias 
 reales del grupo son desconocidas.

A veces, se evalúan las recomendaciones de grupo comparándolas con una valoración media de las preferencias individuales, pero esta aproximación no 
refleja el complejo funcionamiento de un grupo real. Otros métodos comparan las recomendaciones con las preferencias individuales, asumiendo que la satisfacción de 
cada miembro del grupo depende exclusivamente de su propia preferencia, ignorando la influencia de los demás miembros del grupo. A pesar de esto, las 
investigaciones suelen favorecer estrategias que se asemejan a la media.

No obstante, existen estudios que se apartan de la creación de grupos sintéticos, recopilando en cambio datos reales de individuos y de sus elecciones cuando 
están en grupos. Estos enfoques ofrecen una visión más precisa y realista de las dinámicas de grupo en los sistemas de recomendación, pero son más difíciles de
conseguir.


\subsection{Implicaciones para el trabajo\label{SEC:DISCUSION_RECOMENDACION}}

Teniendo en cuenta el contexto de la aplicación que queremos desarrollar, y las opciones que nos da la API de \textit{Spotify}, podríamos tratar 
de clasificar en base a las características expuestas en la Sección \ref{SEC:RECOMENDACION_GRUPOS} el sistema de recomendación que podemos obtener. 
Por ello, en cuanto al que va a ser nuestro sistema de recomendación, podemos decir que:
\begin{itemize}
    \item Conoceremos las preferencias individuales de los usuarios de antemano gracias a los datos de \textit{Spotify}.
    \item Los ítems serán experimentados directamente por el grupo, ya que se trata de música.
    \item Buscamos que el grupo pueda ser activo, pudiendo elegir distintas estrategias de agregación de preferencias.
    \item Nuestro sistema suministrará recomendaciones directamente tras una previa 'negociación'.
    \item Nuestro sistema recomendará secuencias de ítems, las \textit{playlists}.
    \item Las preferencias individuales de los usuarios son estimadas de forma implícita, a través de sus interacciones con \textit{Spotify}.
    \item En el caso de nuestro sistema de recomendación, agregaremos recomendaciones individuales de ítems en una recomendación grupal. No agruparemos
    perfiles de usuarios en un único perfil grupal.
\end{itemize}

En cuanto a las estrategias expuestas en la Sección \ref{SEC:AGREGACION}, la utilización de ellas será fundamental. Tal y como las utilizaremos será para obtener
un \textit{rating} grupal para cada canción que sea susceptible de entrar en la \textit{playlist} que se acabará generando para el grupo. Así que, lo que haremos
básicamente, será obtener recomendaciones de individuales de canciones para cada usuario, y luego, a través de una estrategia de agregación, obtener un 
\textit{rating} grupal para cada canción. Ahondaremos más sobre el funcionamiento e implementación en la Sección \ref{SEC:IMPLEMENTACION}.