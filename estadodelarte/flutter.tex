\textit{Flutter} \cite{flutter} es un \textit{framework} de desarrollo de código abierto de aplicaciones multiplataforma desarrollado 
por \textit{Google}. Es una alternativa que tiene cierta popularidad, y que, como comentábamos en la introducción,
ha llamado nuestra atención.

Con este \textit{framework}, se pueden desarrollar aplicaciones para diversas plataformas, con una misma base de código.
Desde aplicaciones móviles en \textit{iOS} y \textit{Android}, hasta aplicaciones de escritorio para \textit{Windows}, \textit{Linux} o \textit{MacOS}. Sin olvidarnos de
las aplicaciones \textit{Web} (como la aplicación desarrollada en este trabajo), o incluso dispositivos 
embebidos, como los que pueden ser el infoentretenimiento de un coche \cite{flutter_car}. 

Aunque solamente este hecho aporta un atractivo muy considerable a esta tecnología, \textit{Flutter} cuenta con otras características a tener en cuenta:

\begin{itemize}
  \item \textbf{Rendimiento:} utiliza el lenguaje de programación \textit{Dart} \cite{dart}, que se compila a código nativo, lo que aporta una ventaja en rendimiento clara.
   Comentaremos más sobre las características de este lenguaje en la Sección \ref{SEC:DART}.
  \item \textbf{Productividad:} cuenta con \textit{Hot Reload} \cite{hot_reload}, que permite ver los cambios en la aplicación en tiempo real, sin necesidad de reiniciar 
  la aplicación, y sin perder el estado de esta, agilizando mucho el desarrollo.
  \item \textbf{Personalización:} su diseño de la interfaz basada en \textit{widgets} personalizables, que permiten crear interfaces de usuario atractivas y consistentes.
  \item \textbf{Comunidad:} existe una comunidad con buena actividad, la cual hace que prolifere el desarrollo de \textit{plugins} que se pueden encontrar en \textit{pub.dev} \cite{pub_dev}.
\end{itemize}


\subsection{\textit{Dart}, el lenguaje detrás de \textit{Flutter}\label{SEC:DART}}

\textit{Flutter} se basa en el lenguaje de programación \textit{Dart} \cite{dart}, 
también desarrollado por \textit{Google}. Aunque este es un lenguaje relativamente nuevo, y con 
un uso que no está tan extendido como otros lenguajes, como \textit{Java}, \textit{Python} o \textit{JavaScript},
podemos destacar varias características de este lenguaje que lo hacen atractivo. Por otro lado, haberse familiarizado
con este lenguaje no ha sido un problema para el desarrollo en este trabajo, debido a una sintaxis que 
recuerda a otros lenguajes como \textit{Java}, \textit{C++} o \textit{JavaScript}. Algunas características 
para una visión general de \textit{Dart} \cite{dart_overview} son:

\begin{itemize}
  \item \textbf{Tipado estático:} es un lenguaje de tipado estático, lo que significa que una vez las 
  variables se declaran de un tipo, este no puede cambiar durante la ejecución del programa.
  \item \textbf{Tipado dinámico:} aunque sea un lenguaje de tipado estático, también permite el uso del tipo
  \texttt{dynamic}, que permite flexibilidad en el tipo que deba tener una variable.
  \item \textbf{Programación orientada a objetos:} es un lenguaje orientado a objetos, lo que lo hace muy amigable
  si se tiene experiencia con otros lenguajes orientados a objetos.
  \item \textbf{Seguridad frente a nulos:} tiene un sistema de seguridad frente a nulos, que permite evitar que variables
puedan tener un valor nulo en cualquier momento en tiempo de ejecución.
  \item \textbf{Asincronía:} permite el uso de funciones asíncronas, que permiten ejecutar código en segundo plano, 
  sin bloquear la ejecución del programa.
  \item \textbf{Compilación a código nativo:} se compila a código nativo, lo que aporta un rendimiento muy bueno.
  \item \textbf{Compilación nativa:} permite la compilación \textit{JIT} (\textit{just-in-time}) y \textit{AOT} (\textit{ahead-of-time}), 
  que permite producir código máquina para plataformas nativas.
  \item \textbf{Compilación \textit{Web}:} para aplicaciones \textit{Web}, se compila a código
  \textit{JavaScript}, que es lo que ocurre con la aplicación desarrollada en este trabajo.
\end{itemize}

En resumen, Dart es un lenguaje optimizado para el cliente que permite desarrollar aplicaciones rápidas en cualquier 
plataforma. Su objetivo es ofrecer el lenguaje de programación más productivo para el desarrollo 
multiplataforma, junto con una plataforma de ejecución flexible para marcos de aplicaciones. 
Dart también constituye la base de Flutter. Dart proporciona el lenguaje y los tiempos de ejecución 
que impulsan las aplicaciones de Flutter, pero Dart también admite muchas tareas básicas del desarrollador, 
como formatear, analizar y probar el código \cite{dart_overview}.

\subsection{Instalación y primer uso básico\label{SEC:INSTALACION}}

Para poder desarrollar con \textit{Flutter}, es necesario instalar el \textit{SDK} de 
\textit{Flutter}, que se explica en \cite{flutter_install} para diferentes plataformas. Una vez instalado, se puede
comprobar que todo funciona correctamente ejecutando el comando\texttt{flutter doctor}. Para crear un proyecto, 
se ejecuta el comando \texttt{flutter create <nombre\_proyecto>}.
Una vez creado, tras situarse en el directorio del nuevo proyecto, se puede ejecutar
con el comando \texttt{flutter run}. Ejecutando el comando \texttt{flutter devices} se puede
comprobar qué dispositivos están conectados y disponibles para la ejecución. En la Figura \ref{FIG:FLUTTER_RUN} 
se puede ver un ejemplo de ejecución de un proyecto \textit{Flutter}.

\begin{figure}[Ejecución de un proyecto \textit{Flutter}]{FIG:FLUTTER_RUN}
  {Ejecución de un proyecto \textit{Flutter}. \\
   \footnotesize{
    Imagen extraída de \href{
      https://docs.flutter.dev/get-started/test-drive?tab=terminal
    }{flutter.dev}
    }
  }
        \image{6cm}{}{propias/flutter_run.png}
\end{figure}


\subsection{\textit{Widgets}\label{SEC:WIDGETS}}

Los \textit{widgets} son el elemento básico de la interfaz de usuario en \textit{Flutter}. Todo lo que se ve
en pantalla es un \textit{widget}, y estos se combinan para crear elementos gráficos más complejos; pero 
siempre hay que tener en mente que todo está formado por \textit{widgets}, los cuales se organizan en forma de
árbol. En la Figura \ref{FIG:WIDGETS} se puede ver un ejemplo de vista de árbol de \textit{widgets}.

\begin{figure}[Ejemplo de árbol de \textit{widgets}]{FIG:WIDGETS}
  {Ejemplo de árbol de \textit{widgets}.}
        \image{6cm}{}{propias/widget_tree.png}
\end{figure}

Según \cite{flutter_ui}, los widgets de Flutter se construyen utilizando un marco moderno que se inspira en \textit{React} \cite{react}. 
La idea central es construir la interfaz de usuario a partir de widgets. Los widgets describen el aspecto que debería 
tener su vista en función de su configuración y estado actuales. Cuando cambia el estado de un widget, este reconstruye 
su descripción, que el framework compara con la descripción anterior para determinar los cambios mínimos necesarios en el 
árbol de renderizado subyacente para pasar de un estado al siguiente.

Existen dos tipos de \textit{widgets}: los \textit{StatelessWidget} y los \textit{StatefulWidget}. Los primeros son aquellos que no contemplan
cambios en su estado porque no tienen; es decir, no almacenan un estado que pueda variar. Mientras que los segundos sí almacenan un estado que pueda
variar, y por ejemplo, pueden cambiar su aspecto en función de este estado.

Para hacer ver esto de una forma más cercana a lo técnico y al código, vamos a aclarar un par de cosas. Cuando se crea un \textit{widget}, se define una clase
que hereda de \textit{StatelessWidget} o de \textit{StatefulWidget}, y se implementa el método \texttt{build()}. Este método es el que se encarga de devolver
un \textit{widget} (que, como ya hemos comentado, puede estar formado por numerosos \textit{widgets} estructurados en un árbol) que será lo que se muestre en 
la interfaz de usuario.

En el caso de los \textit{StatefulWidget}, debe implementarse el método \texttt{createState()}, que devolverá un objeto que hereda de \textit{State}.
Este será el que se encargue de almacenar el estado del \textit{widget}, y actualizarlo cuando se necesite. En el caso de un \textit{StatefulWidget}, el método
\texttt{build()} puede implementarse tanto en el \textit{widget} como en el \textit{state}, pero es recomendable hacerlo en el \textit{state}.
Veamos la comparación en código de un \textit{StatelessWidget} y un \textit{StatefulWidget} en la Figura \ref{FIG:WIDGETS_CODE}.

\begin{figure}[Implementación en código de un \textit{StatelessWidget} y un \textit{StatefulWidget}]{FIG:WIDGETS_CODE}
  {Implementación en código de un \textit{StatelessWidget} y un \textit{StatefulWidget}.}
  \subfigure[]{\textit{StatelessWidget}}{\image{7cm}{}{propias/stateless.png}} \quad
  \subfigure[]{\textit{StatefulWidget}}{\image{7cm}{}{propias/stateful.png}}
\end{figure}



