Para este trabajo se tiene como objetivo principal el desarrollo de una aplicación web que permita a grupos de usuarios de \textit{Spotify}
crear \textit{playlists} que combinen sus gustos musicales, ofreciendo una alternativa a las \textit{playlists} que la aplicación de \textit{Spotify}
 les pueda ofrecer. Tras ello, queremos llevar a cabo el estudio y posterior comparación de distintas estrategias de agregación para la generación de \textit{playlists}.
 
 Derivados de este objetivo principal, se plantean los siguientes objetivos específicos:

\begin{itemize}
  \item Estudiar sobre el desarrollo de aplicaciones con \textit{Flutter}, y familiarizarse con la base de su desarrollo, los \textit{widgets}.
  \item Estudiar la API de \textit{Spotify}, leyendo su documentación y haciendo pruebas, para acabar desarrollando el módulo de llamadas a la API.
  \item Desarrollar la aplicación web, que consistirá en un \textit{frontend} que se sirva de la API de \textit{Spotify}, que permita a grupos de 
  usuarios de \textit{Spotify} crear \textit{playlists} que combinen sus gustos musicales.
  \item Realizar un pequeño estudio sobre la aplicación desarrollada, llevando a cabo pruebas con usuarios reales, para obtener conclusiones sobre 
la usabilidad de la aplicación y las potenciales estrategias para la generación de \textit{playlists}.
\end{itemize}
