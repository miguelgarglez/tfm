La motivación de este trabajo viene por dos vertientes principales. Por un lado,
\textit{Flutter} \cite{flutter} es un \textit{framework} de desarrollo de aplicaciones multiplataforma
bastante nuevo, desarrollado por \textit{Google} que ha ganado cierta popularidad
 y que ha llamado nuestra atención. Así que hacer un proyecto con él es una buena 
 forma de ponerlo a prueba y ver qué tal funciona.

 Por el otro lado, escuchar música forma parte de nuestro día a día, de manera directa,
 a través de la aplicación de \textit{Spotify} \cite{spotify}. Añadido a esto, percibimos
 que hoy en día en muchas aplicaciones, productos o servicios, el usuario tienen un papel
 muy pasivo, y que en el caso de \textit{Spotify} esto se ha acentuado con el paso del tiempo.
Los usuarios reciben distintos tipos de \textit{playlists} generadas periódicamente: novedades de la semana, 
fusionadas con otros usuarios que se actualizan semanalmente, predeterminadas para estados de ánimo, etc.
Por supuesto, los usuarios pueden crear sus propias \textit{playlists}, pero la forma de hacerlo es bastante manual, y desde luego
la adición de canciones se hace de una en una.

 Por ello, planteamos explorar una alternativa intermedia, en la cual, usuario y sistema comparten el peso de
 la creación de las \textit{playlists}. Podrán juntarse varios usuarios a la vez para crear una
 \textit{playlist} que trate de ajustarse a los gustos de todos. Incluso, generar una \textit{playlist} 
 que se base en sus, por ejemplo, tres canciones más escuchadas recientemente. La principal motivación es ofrecer al 
 usuario, y sobre todo a aquel que le guste cobrar un papel más activo en sus elecciones, una forma fácil 
 de obtener canciones a escuchar.

 Finalmente, en este trabajo entra en escena la recomendación, y más específicamente, la dirigida a grupos,
 ya que se buscará también hacer un pequeño estudio sobre las distintas formas de generar \textit{playlists} a partir de los
 gustos de un grupo de usuarios. Los experimentos e investigación recaerán principalmente en las distintas estrategias de agregación, las cuales 
 se diferencian en cómo tienen en cuenta las preferencias individuales de los usuarios para generar una \textit{playlist} que se ajuste a todos.
Se utilizará la API de \textit{Spotify} \cite{spotify_api} para obtener datos de usuarios, 
 canciones más escuchadas, y recomendaciones, que aportarán la base para la generación de \textit{playlists}, ya que, a partir de ellas, aplicando
 distintas estrategias de agregación, se obtendrán las \textit{playlists} finales.