El documento seguirá la siguiente estructura:

En primer lugar, en el Capítulo \ref{CAP:ESTADODELARTE} se hará un primer recorrido por el contexto y la importancia de los actores principales en el proyecto, los cuales serán
\textit{Flutter} \cite{flutter}, \textit{Spotify} \cite{spotify} y los sistemas de recomendación \cite{recommender_handbook}.

En el Capítulo \ref{CAP:DESARROLLO} se expondrá todo el proceso de desarrollo de la aplicación, desde el análisis 
de la misma, pasando por el diseño y la implementación final.

En el Capítulo \ref{CAP:PRUEBAS} se expondrán las pruebas realizadas, tanto \textit{offline} como con usuarios, 
los resultados obtenidos, y sus conclusiones e influencia en el proyecto.

Por último, en el Capítulo \ref{CAP:CONCLUSIONES} se expondrán las conclusiones obtenidas, posibles limitaciones que hayamos podido encontrarnos
durante el transcurso del proyecto y se propondrá trabajo futuro.

En los Anexos se incluirán las preguntas de la encuesta a los usuarios, en el Apéndice \ref{CAP:PREGUNTAS_FORMULARIO}, también existe un repositorio público en 
\textit{GitHub} con el código de la aplicación desarrollada cuyo enlace se encuentra en el resumen, y en el
Capítulo \ref{CAP:CONCLUSIONES}.

Además, la teoría sobre recomendación de este trabajo se ha basado practicamente en su totalidad en el libro 
\textit{Recommender Systems Handbook} \cite{recommender_handbook}, desde las ideas básicas sobre sistemas de recomendación, 
hasta las estrategias de agregación que se han utilizado en el proyecto.
